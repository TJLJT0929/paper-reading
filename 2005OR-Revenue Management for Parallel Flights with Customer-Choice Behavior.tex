\begin{document}


\section{Revenue Management for Parallel Flights with Customer-Choice
Behavior}\label{revenue-management-for-parallel-flights-with-customer-choice-behavior}

Dan Zhang, William L. Cooper Department of Mechanical Engineering,
University of Minnesota, 111 Church Street S.E., Minneapolis, Minnesota
55455 \{zdan@me.umn.edu, billcoop@me.umn.edu\}

We consider the simultaneous seat- inventory control of a set of
parallel flights between a common origin and destination with dynamic
customer choice among the flights. We formulate the problem as an
extension of the classic multiperiod, single- flight ``block demand''
revenue management model. The resulting Markov decision process is quite
complex, owing to its multidimensional state space and the fact that the
airline's inventory controls do affect the distribution of demand. Using
stochastic comparisons, consumer- choice models, and inventory- pooling
ideas, we derive easily computable upper and lower bounds for the value
function of our model. We propose simulation- based techniques for
solving the stochastic optimization problem and also describe heuristics
based upon an extension of a well- known linear programming formulation.
We provide numerical examples.

Subject classifications: dynamic programming/optimal control:
applications; transportation: yield management. Area of review:
Manufacturing, Service, and Supply Chain Operations. History: Received
April 2003; revisions received December 2003, May 2004; accepted May
2004.

\section{1. Introduction}\label{introduction}

Airline revenue management involves dynamically controlling the
availability and prices of many different classes of tickets to maximize
revenues. Despite the vast technical literature on the subject of
revenue management, relatively few papers explicitly model the
interaction between the random demand for tickets and the seller's
choice of booking policy. In fact, the majority of work assumes that the
distribution of demand for the various ticket types is exogenously
determined, and is therefore not affected by the booking policy (of
course, all models do assume that the policy does affect which ticket
requests are accepted). Among the papers that do attempt to capture
behavioral effects, most are concerned with how the customers decide
which type of ticket to buy for a single specified flight, given the
availability of the various tickets for the particular flight in
question.

In this paper, we consider a different issue, namely, how to select
booking limits when the airline has many flights between a particular
origin and destination within a short timespan. A check of airline
schedules (using, e.g., online services such as Orbitz or Travelocity)
shows that this scenario occurs frequently in the real world. We focus
on the situation in which customers select among these ``parallel''
flights. In particular, we address what Belobaba (1989) calls
``horizontal shifts'' of passengers to different flights but the same
fare class.

To see the importance of this scenario, consider an airline that has two
flights per day between cities A and B. Suppose that the airline sells
two classes of tickets (high fare and low fare) on each flight, that
nearly all high- fare customers want to fly on the flight later in the
day, and that they are inflexible regarding this preference. Suppose
that, on the other hand, low- fare customers are essentially indifferent
between the flights and are willing to fly on either one. In such a
case, the airline should shut off low- fare ticket sales on the later
flight while keeping many low- fare tickets available for the earlier
one. This will allow the airline to sell tickets to low- fare customers
on the early flight without displacing high- fare customers on the later
flight. Note that this is a different issue than that mentioned above,
whereby customers select among different ticket types on a particular
flight. Alternatively, as described in Karaesman and van Ryzin (2004),
the airline might oversell the earlier flight and bump (at a cost)
customers to the later flight. We do not consider such methods in this
paper.

Currently available models do not provide a formal method for addressing
problems where customers choose among parallel flights. Our Markov
decision process (MDP) modeling approach to such problems is an
extension of that in which there are sequentially realized ``blocks'' of
demand for each ticket class. We employ a vector- valued state variable
(with one entry per flight), and for any given period (i.e., block) we
allow the distribution of sales for each flight to depend upon the
booking limits in effect for all flights. This setup allows for fairly
general customer dynamics within a period.

Because there is quite a large body of work on revenue management, we
limit the scope of our literature review to variants of the expected
marginal seat revenue (EMSR) model, which employ the aforementioned
block- demand setup. We will also describe some recent work that

explicitly models customer choice behavior in various settings. For an
overall survey of the field of revenue management, we refer the reader
to McGill and van Ryzin (1999) or Boyd and Bilegan (2003).

\section{1.1. Literature Review}\label{literature-review}

In some of the earliest revenue- management work, Littlewood (1972)
describes a simple technique for setting a booking limit on the number
of low- fare tickets available for sale in a problem with two fare
classes for a single- leg flight. Later, Belobaba (1989) considers the
case with multiple (more than two) booking classes. Subsequently, a
number of authors have developed a framework for determining booking
limits for a single- leg flight with \(m\) mutually independent demand
classes that arrive in sequential blocks. Hereafter, we will term models
that employ variations of the sequential- block- of- demand assumption
as EMSR- type models. Although the block- demand assumption is not
entirely realistic, these models have, in fact, proved to be useful
ingredients in real- world revenue management systems.

Li and Oum (2002) discuss the rough equivalence of the optimality
conditions of Brumelle and McGill (1993), Curry (1990), and Wolmer
(1992), who all derive optimal policies for EMSR- type models under
slightly varying assumptions. Robinson (1995) extends the analysis to
blocks of arrivals with fares that need not be monotonically increasing.
Several papers consider the interaction between stochastically dependent
booking classes. For instance, Brumelle et al.~(1990) consider two
possibly dependent booking classes, and derive optimality conditions
under mild assumptions on the dependence structure. Cooper and Gupta
(2005) employ stochastic order relations to investigate, among other
things, the effect of stochastic dependence on total expected revenue.

Lautenbacher and Stidham (1999) model the booking control problem on a
single- leg flight as an MDP. They establish the connection between
EMSR- type models and other one- arrival- per- period MDPs. Their
analysis confirms that for a wide class of seat management problems, a
booking- limit policy is indeed optimal. (Implicit in some earlier work
is the consideration of only booking- limit policies. Under more general
models of the customer arrival process, there typically will be non-
booking- limit policies that outperform any booking- limit policy---see,
e.g., Chatwin (1998). Nevertheless, booking- limit policies continue to
dominate airline practice because many distribution channels allow only
these types of policies. Consequently, one could reasonably argue that
the formulation of the problem should indeed require the selection of a
booking- limit policy.) In the model described in this paper, we
consider only booking- limit policies.

As mentioned earlier, relatively few papers consider customer- choice
behavior in the revenue management literature. Talluri and van Ryzin
(2004) analyze an MDP formulation of a single- leg problem with customer
choice among the open fare classes. They show that an optimal policy can
be found by searching over a relatively small class of policies, and
provide conditions under which there is an optimal nested booking- limit
policy for the problem with customer choice. Zhao and Zheng (2001)
consider a two- class single- flight model with flexible customers, and
formulate the problem as an optimal stopping model. Belobaba and
Weatherford (1996) propose corrections to the EMSR heuristics in the
presence of customer diversion. The abovementioned paper of Brumelle et
al.~(1990), in which dependent demands are analyzed, also models the
``sell- up'' phenomenon, whereby closing Class- 2 ticket sales may cause
some would- be Class- 2 customers to purchase Class- 1 tickets. Hence,
the booking policy does induce changes in demand distribution for Class
1. This portion of their study formalizes an earlier analysis of
Belobaba (1989). Other papers that model diversion from one class to
another include Pfeifer (1989) and Bodily and Weatherford (1995).

Several recent papers in the inventory control literature consider
customer- choice behavior in retail situations. Mahajan and van Ryzin
(2001b) analyze the problem of optimizing inventory levels of
substitutable products under dynamic consumer substitution. Smith and
Agrawal (2000) study the effect of substitution on demand and address
the issue of jointly setting stocking levels under a Markovian choice
model. In these models where dynamic customer behavior is taken into
account, exact analysis has proved to be difficult, causing authors to
seek effective approximation methods.

Shumsky and Zhang (2004) consider a multiperiod capacity allocation
model with multiple classes and service upgrading. In their model,
inventory is differentiated (in terms of product or service quality),
and the demand of a lower- class customer may be satisfied by a product
that is one level higher in each period. In contrast to our work, they
assume that prices of different products are stationary across periods,
and allow the allocation in each period to be made after observing
demand in that period. Their main focus is how much unsatisfied lower-
class demand should be upgraded (or alternatively, how much higher-
class inventory can be allocated to lower- class demand).

Netessine and Shumsky (2004) formulate a game- theoretic version of the
single- leg airline revenue management problem assuming two airlines are
in competition, and the demand of one airline depends on the booking
policy of the other airline. They provide conditions under which a Nash
equilibrium (in booking limits) exists under assumptions on consumer
substitution patterns. They also consider a centralized model where the
two flights are assumed to be owned by one airline. Their centralized
model is somewhat similar in spirit to the model considered in our
paper, although their focus is to compare the system- optimal controls
from the centralized solution with the competitive outcome.

Many MDP models, including the one considered here, are too complex for
exact solution or even for storage of

an optimal policy. Consequently, it is of particular interest to develop
methods to solve MDPs approximately---and this is the approach we take
in this paper. For an overview of approximation methods for MDPs, see
Bertsekas and Tsitsiklis (1996). A slightly different idea is that of
Müller (1997), who investigates how the value function of an MDP changes
when the transition probability distributions are changed and all the
other model parameters remain the same. He employs stochastic order
relations to establish the monotonicity of value functions of MDPs with
respect to changes in transition probabilities. This allows one to
generate bounds on the value function of the MDP; however, he does not
discuss using such bounds as an ingredient in a computational procedure.
White and Schlussel (1981) present bounds and approximation procedures
for so- called multimodule MDPs, which possess vector state spaces for
which each element of the vector evolves independently, and each module
is interconnected only through the cost structure. The problem we
consider in this paper does not fall into this category. Lovejoy (1986)
considers a slightly different problem, and proposes several approaches
to generate bounds for the optimal policy.

\section{1.2. Overview of Results and
Outline}\label{overview-of-results-and-outline}

In this paper, we consider the seat- inventory control of multiple
parallel single- leg flights in the presence of dynamic customer- choice
behavior. The objective is to maximize the expected total revenue over
all the flights. We formulate the problem as an MDP.

In our MDP formulation, we assume that the demand in each period on each
flight depends on the numbers of open seats on all the flights. We then
derive separable upper and lower bounds using stochastic comparison
results and dynamic programming principles. We adopt a customer- choice
model that assumes that the choice outcome of each customer is
determined by a preference mapping together with the inventory
availability. Within a fixed period, the choice model we employ is
equivalent to the one developed by Mahajan and van Ryzin (2001a, b).
However, we emphasize a preference ordering as the starting point,
whereas their starting point is a utility- maximization assumption. One
difference is that we consider a multi- period model; whereas Mahajan
and van Ryzin study a single- period model (in a different problem
context).

The bounds we develop, however, are not limited to the particular choice
model, and are applicable to other settings, as long as certain
assumptions are satisfied. In addition, we develop another upper bound
using inventory- pooling ideas. Because the MDP is computationally
intractable, we propose several solution approaches for our model and
test them with numerical examples. Some of these approaches are based
upon value- function approximations derived from the upper and lower
bounds. Others involve a modification of a well- known linear
programming formulation for network revenue management problems. Broadly
speaking, the solution techniques involve simulation- based methods for
approximately solving high- dimensional MDPs.

In summary, the primary contributions of this paper are (1) formulation
of the revenue management problem for parallel flights in the presence
of customer- choice behavior, (2) development of upper and lower bounds
for the value function of the MDP, (3) description of linear-
programming- based heuristics, (4) proposal of simulation- based
solution techniques for the problem, and (5) discussion of numerical
work that shows the proposed approaches are promising.

The remainder of the paper is organized as follows: Section 2 provides
the basic formulation. Section 3 develops upper and lower bounds for the
value function. Section 4 describes static and dynamic booking- limit
policies. Section 5 introduces another upper bound from inventory
pooling. Section 6 describes a choice model and relates it to the
developments in §82--5. Section 7 describes Some approximate solution
procedures. Section 8 provides the linear programming formulation.
Section 9 includes numerical results. Section 10 contains closing
remarks.

\section{2. Markov Decision Process
Formulation}\label{markov-decision-process-formulation}

Throughout, we use superscripts to denote components of a vector and
subscripts to denote time; \(\epsilon^i\) is a vector whose \(i\) th
element is one and all other elements are zeros, and \(\epsilon^0\) is a
vector with all zeros. The dimensions of vectors should be evident from
the context.

We consider the booking control of multiple parallel single- leg flights
between a common origin and destination. Let \(N = \{1,\ldots ,n\}\) be
the set of \(n\) flights. The initial capacity of flight \(i\) is
\(c^i\) ; \(i = 1,\ldots ,n\) . Let \(c\) denote the \(n\) - vector with
entries \(c^i\) . The seats on each flight can be sold to \(m\) possible
classes. Note that we assume that the number of fare classes is the same
on all flights. The fare for class- \(j\) is \(f_j\) . The \(m\) classes
arrive in distinct time periods, with class- \(j\) demand arriving in
period \(j\) . If one wants to allow, say, class \(Y\) to arrive both
before and after class \(Z\) , then we simply relabel the classes as
\(Y,Z\) , and \(Y'\) to put the problem into this context. Therefore, if
an airline has 10 actual booking classes, then \(m\) could be larger
than 10. As we move closer to the time of departure, the time index
decreases, so the first time period is \(t = m\) , and the last is
\(t = 1\) .

Within a time period, we assume that customers choose among the flights
(or decide not to purchase) dynamically according to some choice model;
however, we delay the description of any particular choice model to §6,
because such specific information is not needed at this point. We do not
consider the situation where customers select what class of ticket to
purchase---rather, our model takes class to be exogenous. Although one
could certainly object to this assumption, it is important to point out
that most existing models also make the same assumption. Those that do
allow choice behavior for class typically do not allow choice among
flights. Belobaba (1989) briefly alludes to

such horizontal shifts of passengers to different flights, but the same fare class. However, he does not describe solution approaches for the problem.

We formulate the booking control problem as an MDP. At time \(t\) a
state \(s\) represents the number of seats sold prior to time \(t\)
.Here, \(s\) is an \(n\) - vector, and the ith component \(s^i\) is the
number of seats sold prior to time \(t\) on flight \(i\) .For each
period, we must select an action. The actions we consider are the number
of seats to be open for sale on each flight at the beginning of the
period. Let \(\mathcal{X}\) be the integer) \(n\) - vector of the
numbers of open seats in period \(t\) .We require that
\(0\leqslant x^{i}\leqslant c^{i} - s^{i}\) for \(i = 1,\dots ,n\)
.Formally, the action space for state \(s\) is given by
\(\{x\in \mathbb{Z}^n\colon 0\leqslant x\leqslant c - s\}\) where the
inequalities should be interpreted componentwise. (Throughout,
inequalities involving vectors should be interpreted componentwise.)
Sometimes we will refer to an action \(\mathcal{X}\) as an initial
availability vector for period \(t\)

Let \(Q_{t}^{i}\) be the integer- valued (random) demand for flight
\(i\) in period \(t\) shortly, we shall say more precisely what this
means. Assume that conditional upon action \(\mathcal{X}\) being
selected in period \(t\) ,the random \(n\) - vector \(Q_{t}\) is
independent of the ``past history'' of the process. However, the
conditional distribution of \(Q_{t}\) given an action \(\mathcal{X}\)
does depend upon \(\mathcal{X}\) .Let \(F_{x}^{t}(\cdot)\) denote this
\(n\) - dimensional conditional distribution function. In other words,
if for \(i = 1,\dots ,n\) we make \(x^{i}\) seats available for flight
\(i\) in period \(t\) , then the vector of demand will have conditional
distribution \(F_{x}^{t}(\cdot)\) .At this juncture, it is important to
note that by allowing this distribution to depend upon \(\mathcal{X}\)
we are allowing demand to depend upon seat availability in a rather
general manner. As we pointed out earlier, few revenue management models
have previously allowed any such interactions.

Let \(X_{t}\) denote the action selected for period \(t\) .To simplify
developments below, let \(Q_{t}(x)\) denote a random vector with
distribution \(F_{x}^{t}(\cdot)\) ;i.e.,
\(\mathrm{P}(Q_t\leqslant q\mid X_t = x) =\)
\(\mathrm{P}(Q_t(x)\leqslant q) = F_x^t (q)\) .Note that \(Q_{t}(x)\)
the demand for flight \(i\) when the action is \(\mathcal{X}\) depends
not only on the number of seats open for flight \(i\) but also on the
availabilities for the other flights as well. In addition, we do allow
individual components, say \(Q_{t}^{i}(x)\) and \(Q_{t}^{i}(x)\) , to be
dependent- this allows us to model the interactions of availability and
demand across flights. For a generic function \(g\) we shall use the
notation \(\operatorname {E}[g(Q_t(x))]\) for
\(\begin{array}{r}\int_qg(q)dF_x^t (q) = \end{array}\)
\(\begin{array}{r}\sum_qg(q)\mathrm{P}(Q_t(x) = q) \end{array}\)

When action \(\mathcal{X}\) is selected in period \(t\) and
\(Q_{t} = q\) the period- \(t\) revenue on flight \(i\) is

\[
r_t^i (x^i,q^i) = f_t\min \{x^i,q^i\} , \tag{1}
\]

where \(\min \{x^i,q^i\}\) is the sales for flight \(i\) in period \(t\)
.The total one- period revenue is

\[
r_t(x,q) = \sum_{i = 1}^n r_t^i (x^i,q^i). \tag{2}
\]

Likewise, the expected revenue in period \(t\) conditional upon the use
of action \(\mathcal{X}\) in period \(t\) is given by
\(\operatorname {E}[r_t(x,Q_t(x))]\)

For each \(i\) \(Q_{t}^{i}(x)\) can be interpreted as the number of
sales that would result for flight \(i\) if there were infinitely many
seats available for flight \(i\) and \(x^{j}\) seats available for each
flight \(j\neq i\) .This notion is somewhat related to so- called
unconstrained demand, which can be roughly defined as sales that would
accrue if there were enough capacity on flight \(i\) to satisfy all
incoming customers. Within a period the dynamics whereby a choice of
action \(\mathcal{X}\) combines with ``randomness'' to yield a
realization of \(Q_{t}(x)\) can be quite complicated- this is the
subject of {\$\textbackslash S 6\$}

One might argue that it is more desirable to use the sales in period
\(t\) as the basic random quantity, rather than to go through
\(Q_{t}(x)\) , because \(Q_{t}(x)\) may be hard to identify in practice.
That is, instead of having a quantity \(Q_{t}(x)\) one could start by
defining the single- period revenue to be
\(\begin{array}{r}\mathrm{E}[\sum_{i = 1}^{n}f_{t}Y_{t}^{i}(x)] \end{array}\)
, where the basic random quantity is a sales vector \(Y_{t}(x)\) that
should, of course, satisfy \(\mathrm{P}(0\leqslant\)
\(Y_{t}(x)\leqslant x) = 1\) . However, any such problem can be
transformed into our framework, because
\(Y_{t}(x) = \min \{x,Y_{t}(x)\}\) componentwise under the assumption
\(\mathrm{P}(0\leqslant Y_t(x)\leqslant x) = 1\) In other words, if one
wants to define \(Q_{t}(x)\) to be sales rather than demand, it will be
consistent with our mathematical setup. We take \(Q_{t}(x)\) as
described above as a starting point for our model, because it gives us a
simple correspondence between the model with and without choice
behavior. Note that without choice behavior,
\(Q_{t}^{i}(x) = Q_{t}^{i}\) is indeed what is typically called the
demand for flight \(i\) in period \(t\)

A policy \(\pi\) prescribes the number of seats to be open on each
flight in each period as a function of the booking history up to the
time in question. We shall, without loss of optimality, restrict our
attention to Markovian policies; i.e., those policies that base
decisions upon only the current state (see Puterman 1994). Let
\(u_{t}^{\pi}(s)\) denote the expected total revenue from periods \(t\)
to 1 under policy \(\pi\) given the state at the beginning of period
\(t\) is \(s\) .Then,

\[
u_{t}^{\pi}(s) = \mathrm{E}_{s}^{\pi}\biggl [\sum_{k = 1}^{t}r_{k}(X_{k},Q_{k})\biggr ]. \tag{3}
\]

Here \(\mathrm{E}_s^{\pi}\) denotes expectation with respect to the
distribution induced by policy \(\pi\) when the state at time \(t\) is
\(s\) .The objective of the MDP is to maximize the expected total
revenue; i.e., to maximize \(u_{m}^{\pi}(0)\) .Let \(v_{t}(s)\) be the
maximum expected revenue obtainable from periods \(t\) to 1 given that
the state at the beginning of period \(t\) is \(s\) that is,

\[
v_{t}(s) = \max_{\pi}u_{t}^{\pi}(s). \tag{4}
\]

Well- known results from MDP theory (see, for instance, Puterman 1994)
show that the problem can be reduced to that of iteratively solving the
optimality equations

\[
v_{t}(s) = \max_{0\leqslant x\leqslant c - s}\operatorname {E}[r_{t}(x,Q_{t}(x)) + v_{t - 1}(\min \{x,Q_{t}(x)\} +s)], \tag{5}
\]

and \(v_{0}(s) = 0\) .
\(s\in \{0,1,\ldots ,c^1\} \times \dots \times \{0,1,\ldots ,c^n\}\)
Before we proceed, observe that if we take \(n = 1\) and assume that the
distribution of \(Q_{t}(x)\) does not depend

upon \(x\) then we get a model like those reviewed in Lautenbacher and
Stidham (1999) and Li and Oum (2002).

In principle, one could implement an optimal policy by storing, for each
\(s\) and \(t\) a maximizing action from (5). Upon entry into a state
\(s\) at time \(t\) one would then need only look up the appropriate
action. For moderately large \(n\) however, the formulation above is
rendered intractable by the well- known curse of dimensionality.
Consider a case with \(n = 10\) flights of 100 seats each and \(m = 10\)
time periods. For each given time \(t\) and state \(s\) we need to solve
a potentially difficult integer program. However, even if we were able
to compute an optimal policy, storing it in a ``look- up table'' would
require keeping track of \(101^{10} \times 10 \approx 10^{21}\) actions.
On the other hand, if we have 10 separate flights each with 100 seats
and no choice behavior (so there are 10 decoupled problems), then we
need only store \(10 \times 101 \times 10 \approx 10^{4}\) actions
(without exploiting any of the savings from the existence of structured
optimal policies in the decoupled case). So, the inclusion of the
customer- choice aspect moves the problem from computationally ``easy''
to intractable.

A significant portion of the remainder of the paper will be devoted to
coming up with good (but suboptimal) policies that can be both computed
and stored. Sections 3- 6 focus on theoretical aspects of the problem.
For instance, in §3, we derive separable upper and lower bounds for the
value function. To apply the bounds, we need to find certain bounding
sequences for the demands. We show how this can be done for a general
choice model in §6. Sections 7- 9 describe numerical examples and
heuristic solution approaches (some based on ideas from §§3- 6).
Ultimately, we will focus on two general classes of approaches. In one,
we provide methods for determining so- called static booking- limit
policies, which require storage of just a few predetermined parameters
(the booking limits). In the other, we describe techniques for obtaining
easily computable approximations to the value function, which
subsequently allow us to compute actions ``on the fly''---so, there is
no look- up table, but rather the action for a particular state is
computed upon entry into the state in question.

\section{3. Upper and Lower Bounds}\label{upper-and-lower-bounds}

In this section, we derive upper and lower bounds for the value function
(4). To this end, recall that random variable \(X\) is stochastically
smaller than random variable \(Y\) (written \(X \leqslant_{st} Y\) )
means \(\operatorname {Eg}(X) \leqslant \operatorname {Eg}(Y)\) for all
increasing functions \(g\) for which the expectations exist.
Equivalently,
\(\mathrm{P}(X \leqslant x) \geqslant \mathrm{P}(Y \leqslant x)\) for
all \(x\) (see, e.g., Müller and Stoyan 2002). Let integer- valued
random vectors \(\{\overline{Q}_t: t = 1, \ldots , m\}\) (respectively,
\(\{\underline{Q}_t: t = 1, \ldots , m\}\) ) be such that
\((\overline{Q}_1, \ldots , \overline{Q}_m)\) (respectively,
\((\underline{Q}_1, \ldots , \underline{Q}_m)\) ) are independent.
Moreover, suppose that

\[
\begin{array}{rl} & {\underline{Q}_t^i\leqslant_{st}Q_t^i (x)\leqslant_{st}\overline{Q}_t^i\quad \mathrm{for~all~}x\in \{0,1,\ldots ,c^1\}}\\ & {\qquad \times \dots \times \{0,1,\ldots ,c^n\} ,i\in N,t = 1,\ldots ,m.} \end{array} \tag{8}
\]

Note that the distributions of \(\overline{Q}_t\) and
\(\underline{Q}_t\) do not depend upon the action \(x\) . We will
consider two separate MDPs for each flight using demand sequences
\(\{\overline{Q}_t: t = 1, \ldots , m\}\) and
\(\{\underline{Q}_t: t = 1, \ldots , m\}\) . We follow the time
convention and fare structure as in §2. In subsequent sections, we will
describe methods to construct \(\{\overline{Q}_t: t = 1, \ldots , m\}\)
and \(\{\underline{Q}_t: t = 1, \ldots , m\}\) for particular choice
models.

Let \(\bar{v}_i^i (s^i)\) be the maximum expected revenue obtainable
from periods \(t\) to 1 on flight \(i\) where the demand in period \(k\)
is \(\overline{Q}_k^i\) for \(k = 1, \ldots , t\) and the number of
seats sold before time \(t\) is \(s^i\) . The MDP optimality equation
for flight \(i\) is

\[
\begin{array}{r}\bar{v}_t^i (s^i) = \underset {0\leqslant x^i\leqslant c^i -s^i}{\max}\mathrm{E}\big[r_t^i (x^i,\overline{Q}_t^i) + \bar{v}_{t - 1}^i (\min \{x^i,\overline{Q}_t^i\} +s^i)\big],\\ t = 1,\ldots ,m, \end{array} \tag{6}
\]

and \(\bar{v}_0^i (s^i) = 0\) ; \(s^i \in \{0, 1, \ldots , c^i\}\) .
Similarly, let \(\underline{v}_t^i (s^i)\) be the maximum expected
revenue obtainable from periods \(t\) to 1 on flight \(i\) where the
demand in period \(k\) is \(\underline{Q}_k^i\) for
\(k = 1, \ldots , t\) and the number of seats sold before time \(t\) is
\(s^i\) . The MDP optimality equation for flight \(i\) is

\[
\underline{v}_t^i (s^i) = \max_{0\leqslant x^i\leqslant c^i -s^i}\mathrm{E}\big[r_t^i (x^i,\underline{Q}_t^i) + \underline{v}_{t - 1}^i (\min \{x^i,\underline{Q}_t^i\} +s^i)\big],
\]

\[
t = 1,\ldots ,m, \tag{7}
\]

and \(\underline{v}_0^i (s^i) = 0\) ; \(s^i \in \{0, 1, \ldots , c^i\}\)
.

We will make use of the following result, which is a special case of
Proposition 4 in Cooper and Gupta (2005), several times throughout the
remainder of the paper.

PROPOSITION 1. Consider a single- flight MDP with no choice behavior,
(one- dimensional) demand sequence \(\{D_t: t = 1, \ldots , m\}\) , and
seat capacity \(\kappa\) . Let \(w_t(\cdot)\) be the associated one-
dimensional value function. That is, \(w_t(\cdot)\) satisfies
\(w_t(s) = \max_{0 \leqslant x \leqslant \kappa - s} \mathrm{E}\big[f_t \min \{x, D_t\} + w_{t - 1}(\min \{x, D_t\} + s)\big]\)
. Similarly, consider a single- flight MDP with no choice behavior,
demand sequence \(\{D_t^i: t = 1, \ldots , m\}\) , and capacity
\(\kappa\) , with associated value function \(w_t^i (\cdot)\) . Assume
that demand distributions do not depend upon the actions selected. If
\(D_t \leqslant_{st} D_t^i\) for all \(t\) , then
\(w_t(s) \leqslant w_t^i (s)\) for all \(t\) and \(s\) .

The following proposition is the main result of this section. As
described in the previous section, the upper and lower bounds given
below are simple to compute.

PROPOSITION 2.
\(\begin{array}{r}\sum_{i = 1}^{n}\underline{v}_t^i (s^i)\leqslant \bar{v}_t(s)\leqslant \sum_{i = 1}^{n}\overline{v}_t^i (s^i) \end{array}\)
for \(t =\) \(1,\ldots ,m\)

Proof. The proof is by induction. For \(t = 1\) , we have

\[
v_{1}(s) = \max_{0\leqslant x\leqslant c - s}\mathrm{E}\sum_{i = 1}^{n}f_{1}\min \{x^{i},Q_{1}^{i}(x)\} ,
\]

and

\[
\begin{array}{rl} & {\sum_{i = 1}^{n}\overline{v}_{1}^{i}(s^{i}) = \sum_{i = 1}^{n}\max_{0\leqslant x^{i}\leqslant c^{i} - s^{i}}\mathrm{E}f_{1}\min \{x^{i},\overline{Q}_{1}^{i}\}}\\ & {\qquad = \max_{0\leqslant x\leqslant c - s}\mathrm{E}\sum_{i = 1}^{n}f_{1}\min \{x^{i},\overline{Q}_{1}^{i}\} ,} \end{array} \tag{8}
\]

\[
\begin{array}{rl} & {\sum_{i = 1}^{n}\underline{v}_{1}^{i}(s^{i}) = \sum_{i = 1}^{n}\underset {0\leq x^{i}\leq c^{i} - s^{i}}{\max}\mathbb{E}f_{1}\min \{x^{i},\underline{Q}_{1}^{i}\}}\\ & {\qquad = \underset {0\leq x\leq c - s}{\max}\mathbb{E}\sum_{i = 1}^{n}f_{1}\min \{x^{i},\underline{Q}_{1}^{i}\} .} \end{array} \tag{9}
\]

The final equalities in (8) and (9) hold because \(\overline{Q}_1\) and
\(\underline{Q}_1\) do not depend on \(\mathcal{X}\) . Because
\(\underline{Q}_1^i\leqslant_{st}\underline{Q}_1^i (x)\leqslant_{st}\underline{Q}_1^i\)
and \(\bar{f}_1\min \{x^i,q_i^i\}\) is increasing in \(q_{1}^{i}\) for
all \(x^{i}\) , it follows that
\(\begin{array}{r}\sum_{i = 1}^{n}\underline{v}_{1}^{i}(s^{i})\leqslant \upsilon_{1}(s)\leqslant \sum_{i = 1}^{n}\bar{v}_{1}^{i}(s^{i}) \end{array}\)
, thereby completing the base case.

For the inductive step, we assume for \(t \geq 2\) that

\[
\sum_{i = 1}^{n}\underline{v}_{t - 1}^{i}(s^{i})\leqslant v_{t - 1}(s)\leqslant \sum_{i = 1}^{n}\bar{v}_{t - 1}^{i}(s^{i})
\]

for all \(s\) . Fix \(s\) and let

\[
\tilde{x}\in \underset {0\leqslant x\leqslant c - s}{\arg \max}\mathbb{E}\sum_{i = 1}^{n}\big[r_t^i (x^i,Q_t^i (x)) + \bar{v}_{t - 1}^i (\min \{x^i,Q_t^i (x)\} +s^i)\big].
\]

Then, using the induction hypothesis, we have

\[
\begin{array}{rl} & v_{t}(s) = \underset {0\leq x\leq c - s}{\max}\mathbb{E}\Bigg[\underset {i = 1}{\overset{n}{\sum}}r_{t}^{i}(x^{i},Q_{t}^{i}(x))\\ & \qquad +v_{t - 1}(\min \{x,Q_{t}(x)\} +s)\Bigg]\\ & \leqslant \underset {0\leq x\leq c - s}{\max}\mathbb{E}\Bigg[\underset {i = 1}{\overset{n}{\sum}}r_{t}^{i}(x^{i},Q_{t}^{i}(x))\\ & \qquad +\underset {i = 1}{\overset{n}{\sum}}\underset {i = 1}{\overset{n}{\sum}}(\min \{x^{i},Q_{t}^{i}(x)\} +s^{i})\Bigg]\\ & = \underset {i = 1}{\overset{n}{\sum}}\mathbb{E}[r_{t}^{i}(\tilde{x},Q_{t}^{i}(\tilde{x})) + \bar{v}_{t - 1}^{i}(\min \{\tilde{x}^{i},Q_{t}^{i}(\tilde{x})\} +s^{i})]\\ & \leqslant \underset {i = 1}{\overset{n}{\sum}}\underset {0\leq y\leq c - s}{\max}\mathbb{E}[r_{t}^{i}(y^{i},Q_{t}^{i}(\tilde{x}))\\ & \qquad +\bar{v}_{t - 1}^{i}(\min \{y^{i},Q_{t}^{i}(\tilde{x})\} +s^{i})\Bigg]. \end{array} \tag{10}
\]

Now consider a \(t\) - period one- dimensional MDP without choice
behavior (i.e., with demand independent of action) that has demand
vector \((Q_t^i (\tilde{x}), \bar{Q}_{t - 1}^i, \ldots , \bar{Q}_1^i)\)
. Let \(\tilde{v}_t^i (s^i)\) denote the value function of this MDP.
Observe that the \(i\) th term in the summation (10) is precisely
\(\tilde{v}_t^i (s^i)\) . Proposition 1 now implies that the \(i\) th
term is bounded above by \(\bar{v}_t^i (s^i)\) . Hence,
\(\bar{v}_t(s) \leqslant \sum_{i = 1}^{n} \bar{v}_t^i (s^i)\) . This
completes the proof of the upper bound.

Next, we prove the lower bound. Let

\[
\underline{x}^{i} = \min \big\{0\leqslant k\leqslant c^{i} - s^{i}\colon f_{t}< \underline{v}_{t - 1}^{i}(s^{i} + k) - \underline{v}_{t - 1}^{i}(s^{i} + k + 1)\big\} .
\]

By Theorem 4 and Corollary 2 of Lautenbacher and Stidham (1999),

\[
\underline{x}^{i}\in \underset {0\leqslant x^{i}\leqslant c^{i} - s^{i}}{\operatorname{argmax}}\mathbb{E}\big[r_{t}^{i}(x^{i},\underline{Q}_{t}^{i}) + \underline{v}_{t - 1}^{i}(\min \{x^{i},\underline{Q}_{t}^{i}\} +s^{i})\big],
\]

Note that from (11), the value of \(\underline{x}^i\) is determined by
\(\underline{v}_{t - 1}^i (\cdot)\) , the value of which does not depend
on \(\underline{Q}_t^i\) . Hence, for each \(i\) , \(\underline{x}^i\)
depends upon the distributions of
\((\underline{Q}_{t - 1}^i,\ldots ,\underline{Q}_1^i)\) , but does not
depend upon the distribution of \(\underline{Q}_t^i\) . Consequently,
\(\underline{x}^i\) also maximizes
\(\operatorname {E}[r_t^i (y^i,Q_t^i (\underline{x})) + \underline{v}_{t - 1}^i (\min \{y^i,Q_t^i (\underline{x})\} +s^i)]\)
over \(y^i\in \{0,1,\ldots ,c^i - s^i\}\) . Let \(\bar{v}_t^i (s^i)\)
denote the value function of a \(t\) - period one- dimensional MDP with
no choice behavior that has demand vector
\((Q_t^i (\underline{x}),\underline{Q}_{t - 1}^i,\ldots ,\underline{Q}_1^i)\)
. We now have

\[
\begin{array}{rl} & v_{t}(s) = \underset {0\leqslant x\leqslant c - s}{\max}\mathbb{E}\Bigg[\underset {i = 1}{\overset{n}{\sum}}r_{t}^{i}(x^{i},Q_{t}^{i}(x))\\ & \qquad +v_{t - 1}(\min \{x,Q_{t}(x)\} +s)\Bigg]\\ & \geqslant \underset {0\leqslant x\leqslant c - s}{\max}\mathbb{E}\Bigg[\underset {i = 1}{\overset{n}{\sum}}r_{t}^{i}(x^{i},Q_{t}^{i}(x))\\ & \qquad +\underset {i = 1}{\overset{n}{\sum}}\underset {i = 1}{\overset{n}{\sum}}r_{t - 1}^{i}(\min \{x^{i},Q_{t}^{i}(x)\} +s^{i})\Bigg]\\ & \geqslant \underset {i = 1}{\overset{n}{\sum}}\mathbb{E}\big[r_{t}^{i}(x^{i},Q_{t}^{i}(x)) + \underset {i = 1}{\overset{n}{\sum}}r_{t - 1}^{i}(\min \{x^{i},Q_{t}^{i}(x)\} +s^{i})\big]\\ & = \underset {i = 1}{\overset{n}{\sum}}\hat{v}_{t}^{i}(s^{i})\geqslant \underset {i = 1}{\overset{n}{\sum}}\underset {i = 1}{\overset{n}{\sum}}\hat{v}_{t}^{i}(s^{i}). \end{array}
\]

In the above, the first inequality follows from inductive hypothesis,
and the third inequality follows from Proposition 1. This completes the
proof.

It is also possible to obtain other bounds for \(v_t\) . In §5, we
describe an upper bound based upon inventory pooling. For another lower
bound, define

\[
\begin{array}{rl} & W_{t}^{x}(s) = \mathbb{E}\big[r_{t}(a_{x,s},Q_{t}(a_{x,s}))\\ & \qquad +W_{t - 1}^{x}(\min \{a_{x,s},Q_{t}(a_{x,s})\} +s)\big],\quad t = 1,\ldots ,m, \end{array}
\]

where \(a_{x,s} = \min \{x,c - s\}\) and \(W_0^x (s) = 0\) . Let

\[
W_{t}(s) = \max_{0\leqslant x\leqslant c - s}\mathbb{E}\big[r_{t}(x,Q_{t}(x)) + W_{t - 1}^{x}(\min \{x,Q_{t}(x)\} +s)\big].
\]

It is straightforward to verify that \(W_t(s) \leqslant v_t(s)\) .
Intuitively, \(W_t(s)\) represents the situation where seat availability
remains unchanged (with the exception that we do not oversell capacity)
in periods \(t, \ldots , 1\) . In our experience with small numerical
examples, (12) can be either larger or smaller than the lower bound in
Proposition 2. However, the bounds in Proposition 2 are much easier to
compute. Indeed, the maximization in (12) becomes intractable for large
\(s\) , rendering (12) useless in such cases. In fact, exact evaluation
of \(W_t(s)\) requires roughly the same amount of computational effort
as solving the multidimensional MDP exactly.

\section{4. Booking Limits: Static Versus
Dynamic}\label{booking-limits-static-versus-dynamic}

As described by Lautenbacher and Stidham (1999) and others, there is an
optimal policy for models without customer choice with an appealing
simple form. For our purposes, these results imply that an optimal
policy \(\underline{\pi}^i\) for flight \(i\) of the lower- bound
problem (given by the maximizers in (7)) can be determined by a sequence
\((\underline{b}_1^i,\underline{b}_2^i,\dots,\underline{b}_m^i)\) as
follows:

\[
\pi_t^i (s^i) = (b_t^i -s^i)^+. \tag{13}
\]

That is, if the state is \(s^i\) at time \(t\) for the one- dimensional
problem with demand process \(\{Q_t^i\}\) it is optimal to make
\((\underline{b}_t^i - s^i)^+\) seats available. See, e.g., Lautenbacher
and Stidham (1999) for more on how to compute
\((\underline{b}_1^i,\underline{b}_2^i,\dots,\underline{b}_m^i)\) . The
existence of optimal policies of this type for the standard no- choice
single- leg model allows one to store an optimal policy in terms of just
\(m\) parameters.

In our formulation (4)- (5), we consider only bookinglimit policies;
implementing action \(\mathcal{X}\) in state \(s\) gives a vector of
booking limits of \(s + x\) .We can divide the class of booking- limit
policies into what we shall term static booking- limit policies and
dynamic booking- limit policies. A static booking- limit policy
\(\pi^b\) is a policy of the form

\[
(\pi^b)_t^i (s) = (b_t^i -s^i)^+, \tag{14}
\]

where \(b\) is a matrix whose \((i,t)\) th element is the booking limit
for flight \(i\) in period \(t\) Expression 14) means that at time \(t\)
and state \(s\) the policy \(\pi^b\) makes \((b_t^i - s^i)^+\) seats
available for flight \(i\) \(i = 1,\dots,n\) . The modifier static'
captures the fact that the matrix \(b\) is fixed and determined ahead of
time). Policies that are not of this type we shall term dynamic booking-
limit policies. Using this terminology, we can restate the result above
as follows: Singleleg problems have an optimal static booking- limit
policy. Unfortunately, this is not the case when we have choice
behavior, as the following example shows.

EXAMPLE 1. Consider a deterministic problem with two flights, each of
which has two seats. There are more than two periods. In Period 2, there
is one customer, who is willing to purchase a seat only on Flight 2. In
Period 1, there are two customers, each of whom strictly prefers seats
on Flight 1, but is willing to accept seats on Flight 2. Using the
terminology of {\$\textbackslash S 2\$} we have

\[
\begin{array}{rl} & Q_{2}(x) = (0,1),\\ & Q_{1}(x) = \left\{ \begin{array}{ll}(2,0) & \mathrm{if~}x^{1}\geqslant 2,\\ (2,1) & \mathrm{if~}x^{1} = 1,\\ (2,2) & \mathrm{if~}x^{1} = 0. \end{array} \right. \end{array}
\]

Let the fare in Period 2 be \(f_{2} = 50\) and the fare in Period 1 be
\(f_{1} = 100\) .At the start of Period 2, if there are, respectively,
one and two remaining seats on Flights 1 and 2 (i.e., if the state is
\((1,0))\) then the optimal action in

Period 2 is \(x = (0,1)\) .That is, only Flight 2 will be open and we
will accept at most one booking. If instead the number of remaining
seats is \((0,2)\) i.e., if the state is \((2,0))\) then the optimal
action is \((0,0)\) i.e., both flights should be closed. Hence, the
optimal policy is not a static bookinglimit policy.

The following example shows that the value function is not componentwise
concave (for one- dimensional problems without choice behavior, the
value function is concave).

ExAMPLE 2. Consider the last period of a deterministic problem with two
flights each with seat capacity 3. Suppose that the fare is
\(f_{1} = 100\) . There are two customer arrivals in the period. The
first customer prefers Flight 1, but is willing to accept seats on
Flight 2. The second customer is only willing to accept seats on Flight
1. The order of arrival matters. If the number of remaining seats is
\((0,1)\) i.e.,if the state is \((3,2))\) then the total number of sales
is 1, so \(v_{1}(3,2) = 100\) . If the number of remaining seats is
\((1,1)\) (i.e., if the state is \((2,2))\) then the total number of
sales is 1 and \(v_{1}(2,2) = 100\) . If the number of remaining seats
is \((2,1)\) (i.e., if the state is \((1,2))\) , then the total number
of sales is 2 and \(v_{1}(1,2) = 200\) . Therefore, the value function
is not componentwise concave. Because \(v_{1}(0,2) = 200\) ,it is also
not componentwise convex.

Given the simplicity of static booking- limit policies, it is of general
interest to ask whether or not there are good policies within this
narrow class. The following proposition says that the expected revenue
from the simple static booking- limit policy (13) described above is at
least as good as the optimal expected revenue for the lower- bound
problem. Let \(u_{t}^{\underline{\pi}}(s)\) be the expected revenue from
using the booking limits from the lower- bound problem as an operating
policy for periods \(t,t - 1,\ldots ,1\) in the actual problem; i.e.,
\(\pi_{t}(s) = (\pi_{t}^{1}(s^{1}),\dots,\pi_{t}^{n}(s^{n}))\) see 347
and (13). Let \(v_{t}^{b}(s) = \sup_{b}u_{t}^{\pi^{b}}(s)\) be the
maximum expected revenue possible from a static booking- limit policy.

\[
\begin{array}{r}PROPOSITION3.\sum_{i = 1}^{n}\underline{v}_{t}^{i}(s^{i})\leqslant u_{t}^{\underline{\pi}}(s)\leqslant v_{t}^{b}(s)\leqslant v_{t}(s). \end{array}
\]

ProoF. The third inequality follows immediately from (4). The second
inequality follows immediately from the definition of \(v_{t}^{b}(s)\) .
Therefore, it remains only to prove the first inequality. This is done
by induction. For \(t = 1\)

\[
\begin{array}{rl} & u_{1}^{\pi}(s) = \mathrm{E}\sum_{i = 1}^{n}f_{1}\min \{c^{i} - s^{i},Q_{1}^{i}(c - s)\} \\ & \qquad \geqslant \mathrm{E}\sum_{i = 1}^{n}f_{1}\min \{c^{i} - s^{i},\underline{Q}_{1}^{i}\} = \sum_{i = 1}^{n}\underline{v}_{1}^{i}(s^{i}). \end{array}
\]

For the inductive step, assume that

\[
u_{t - 1}^{\underline{\pi}}(s)\geqslant \sum_{i = 1}^{n}\underline{v}_{t - 1}^{i}(s^{i}).
\]

Let \(\hat{v}_t^i (s^i)\) be the value function of \(t\) - period one-
dimensional MDP with no choice behavior that has demand vector
\((Q_{t}^{i}((b_{t} - s)^{+}),\underline{Q_{t - 1}^{i}},\dots,\underline{Q_{1}^{i}})\)
. Then,

\[
\begin{array}{rl} & {u_t^{\pi}(s) = \mathrm{E}\Bigg[\sum_{i = 1}^{n}f_t\min \{(b_t^i -s^i)^+,Q_t^i ((b_t - s)^+)\}}\\ & {\qquad +u_{t - 1}^{\pi}(\min \{(b_t^i -s)^+,Q_t((b_t - s)^+)\} +s)\Bigg]}\\ & {\qquad \geqslant \sum_{i = 1}^{n}\mathrm{E}[f_t\min \{(b_t^i -s^i)^+,Q_t^i ((b_t - s)^+)\}}\\ & {\qquad +v_{t - 1}^i (\min \{(b_t - s)^+,Q_t((b_t - s)^+)\} +s)]}\\ & {\qquad = \sum_{i = 1}^{n}\hat{v}_t^i (s^i)\geqslant \sum_{i = 1}^{n}v_t^i (s^i).} \end{array}
\]

In the above, the first inequality follows from the induction
hypothesis, and the second inequality follows from Proposition 1.

\section{5. Inventory Pooling}\label{inventory-pooling}

In this section, we describe how to obtain a different upper bound based
upon inventory pooling. Let
\(Y_{t}^{i}(x)\equiv \min \{x^{i},Q_{t}^{i}(x)\}\) be the random sales
made for the ith flight in period \(t\) in the original problem with
seat allocation \(\mathcal{X}\) , and define
\(\begin{array}{r}Y_{t}^{\mathrm{p}}(x)\equiv \sum_{i = 1}^{n}Y_{t}^{i}(x) \end{array}\)
. Observe that
\(\begin{array}{r}Y_{t}^{\mathrm{p}}(x)\leqslant \min \{\sum_{i = 1}^{n}x^{i},\sum_{j = 1}^{n}Q^{i}(x)\} \leqslant \sum_{i = 1}^{n}x^{i} \end{array}\)
. Hence, for each \(\mathcal{X}\) we have

\[
Y_{t}^{\mathrm{p}}(x) = \min \left\{\sum_{i = 1}^{n}x^{i},Y_{t}^{\mathrm{p}}(x)\right\} . \tag{15}
\]

Let \(\scriptstyle c^{\mathrm{p}} = \sum_{i = 1}^{n}c^{i}\) be the total
capacity of the \(n\) flights, and suppose that there exists an integer-
valued random variable \(D_{t}\) that satisfies
\(Y_{t}^{\mathrm{p}}(x)\leqslant_{st}D_{t}\) for all \(\mathcal{X}\)
.The random variable \(D_{t}\) can be interpreted as the number of
customers that are interested in buying a ticket in period \(t\) .In the
next section, we provide a construction consistent with this
interpretation. Let \(v_{t}^{\mathrm{p}}(\cdot)\) be the MDP value
function of the pooled problem, in which we have one flight with
capacity \(c^{\mathrm{p}}\) and independent onedimensional demand
sequence \(\{D_t\}\) .That is, \(v_{t}^{\mathrm{p}}(\cdot)\) satisfies
\(\begin{array}{r}v_{t}^{\mathrm{p}}(k) = \max_{0\leqslant c\leqslant c^{\mathrm{p}} - k}\mathrm{E}[f_{t}\min \{z,D_{t}\} +v_{t - 1}^{\mathrm{p}}(\min \{z,D_{t}\} + \end{array}\)
\(k)]\) ,where \(k\) and \(\mathcal{Z}\) are integer scalars.

PROPOSITION 4. For each \(t\) we have
\(\begin{array}{r}v_{t}^{\mathrm{p}}(\sum_{i = 1}^{n}s^{i})\geqslant v_{t}(s) \end{array}\)
for all \(s\)

Proor. The proof is by induction. The statement holds for \(t = 1\) .For
the inductive step, assume that the result is true for \(t - 1\) ; that
is,
\(\begin{array}{r}v_{t - 1}^{\mathrm{p}}(\sum_{i = 1}^{n}s^{i})\geqslant v_{t - 1}(s) \end{array}\)
for all \(s =\) \((s^{1},\ldots ,s^{n})\) . Next, consider an arbitrary
state vector \(s =\) \((s^{1},\ldots ,s^{n})\) and let
\(\begin{array}{r}s^{\mathrm{p}} = \sum_{i = 1}^{n}s^{i} \end{array}\) .
By the induction hypothesis,

\[
\begin{array}{rl} & v_{t}(s) = \underset {0\leqslant x\leqslant c - s}{\max}\mathrm{E}\Bigg[\sum_{i = 1}^{n}f_{t}Y_{t}^{i}(x) + v_{t - 1}(Y_{t}(x) + s)\Bigg]\\ & \qquad \leqslant \underset {0\leqslant x\leqslant c - s}{\max}\mathrm{E}\big[f_{t}Y_{t}^{p}(x) + v_{t - 1}^{p}(Y_{t}^{p}(x) + s^{p})\big]\\ & \qquad = \mathrm{E}\big[f_{t}Y_{t}^{p}(\tilde{x}) + v_{t - 1}^{p}(Y_{t}^{p}(\tilde{x}) + s^{p})\big], \end{array} \tag{16}
\]

where \(\tilde{x}\) is the maximizer in (16). Relation (15) now implies

\[
\begin{array}{rl} & v_{t}(s)\leqslant \mathrm{E}\Bigg[f_{t}\min \bigg\{\sum_{i = 1}^{n}\tilde{x}^{i},Y_{t}^{\mathrm{p}}(\tilde{x})\bigg\} \\ & \qquad +v_{t - 1}^{\mathrm{p}}\bigg(\min \bigg\{\sum_{i = 1}^{n}\tilde{x}^{i},Y_{t}^{\mathrm{p}}(\tilde{x})\bigg\} +s^{\mathrm{p}}\bigg)\Bigg]. \end{array}
\]

Because \(0\leqslant \tilde{x}^{i}\leqslant c^{i} - s^{i}\) for
\(i = 1,\dots,n\) , it follows that \(0\leq\)
\(\scriptstyle \sum_{i = 1}^{n}\tilde{x}^{i}\leqslant c^{\mathrm{p}} - s^{\mathrm{p}}\)
. Hence, from the above we have

\[
\begin{array}{rl} & v_{t}(s)\leqslant \underset {0\leqslant z\leqslant c^{\mathrm{p}} - s^{\mathrm{p}}}{\max}\mathrm{E}\big[f_{t}\min \{z,Y_{t}^{\mathrm{p}}(\tilde{x})\} \\ & \qquad +v_{t - 1}^{\mathrm{p}}(\min \{z,Y_{t}^{\mathrm{p}}(\tilde{x})\} +s^{\mathrm{p}})\big]\\ & \leqslant \underset {0\leqslant z\leqslant c^{\mathrm{p}} - s^{\mathrm{p}}}{\max}\mathrm{E}\big[f_{t}\min \{z,D_{t}\} \\ & \qquad +v_{t - 1}^{\mathrm{p}}(\min \{z,D_{t}\} +s^{\mathrm{p}})\big]\\ & = v_{t}^{\mathrm{p}}(s^{\mathrm{p}}), \end{array} \tag{18}
\]

where Inequality (18) follows from Proposition 1. Observe that the
maximizations in (17) and (18) are over onedimensional sets. The final
equality above is simply the optimality equation for the pooled problem.

Before we proceed, it is tempting to consider the control strategy
whereby total availability for all flights combined is given by the
maximizer in (18). However, this is not allowed in formulation (5),
because (5) assumes that the distribution of \(Q_{t}(x)\) is induced by
an initial availability \(x\) In fact, the formulation can be extended
in some cases (e.g., when the choice model in {\$\textbackslash S 6\$}
is in place) to allow strategies of this form. However, in several of
our test problems, the method implied by the maximization in (18) does
not perform as well as other heuristics (please refer to
{\$\textbackslash S 9\$} ,where the ``pooled inventory'' policy is
called PBL, for details).

\section{6. A Choice Model}\label{a-choice-model}

In this section, we describe a customer- choice model, and we show how
to derive from it upper and lower bounds on the distribution of
\(Q_{t}(x)\) as needed to apply the results of {\$\textbackslash S 3\$}
.We also identify \(D_{t}\) as needed for Proposition 4. The
developments also show us how to, in principle, derive the distributions
\(F_{x}^{t}(\cdot)\) of \(Q_{t}(x)\) from other more basic quantities
that reflect consumers' underlying preferences. In addition, we provide
a formal description of ``demand'' when consumer choice is affected by
availability.

\section{6.1. The Model}\label{the-model}

We study the customer- choice process in a generic booking period, so we
suppress the subscript \(t\) in this section. We will focus on how the
choice of initial availability vector \(\mathcal{X}\) (recall
\(\mathcal{X}\) is an action) affects customer behavior.

Assume that customers arrive randomly throughout the period. Let
\(N_0 = \{0,1,\ldots ,n\}\) and \(M = \{1,2,\ldots ,n + 1\}\) . The
elements of \(N_0\) represent the different flights, and O denotes

the no- purchase option. A one- to- one mapping
\(\theta \colon N_0\to M\) for which \(\theta (0)\neq 1\) is called a
preference mapping. Each arriving customer is assumed to have such a
mapping, which he uses to decide what (if anything) to purchase. The
preference mapping of a customer is determined before he observes the
inventory availability. For \(k,l\in N_0\) we say that flight \(k\) is
preferred to \(l\) if \(\theta (k)< \theta (l)\) .The requirement
\(\theta (0)\neq 1\) means that no customer's most preferred option is
not to purchase a ticket. The requirement that a preference mapping is
one to one ensures that each customer has strict preference order among
the flights (and the no- purchase option). This model is essentially
equivalent to ones that use random utility maximization as a starting
point (see, e.g., Mahajan and van Ryzin 2001b, {\$\textbackslash S
2.2.1)\$}

Let \(D\) be an almost surely finite nonnegative integervalued random
variable that represents the total number of customer arrivals in period
\(t\) .For each \(k\) ,let \(\Theta_k\) denote the random preference
mapping of the kth arriving customer. The basic random quantity is
\((D,\{\Theta_k\colon k\in \mathbb{Z}^+\})\) .We assume that the
distribution of \((D,\{\Theta_k\colon k\in \mathbb{Z}^+\})\) is not
affected by the inventory policy or the initial condition at the
beginning of period \(t\) . The distributions of these vectors in
different time periods are independent (this is needed for us to have an
MDP). However, for a fixed period \(t\) we allow a completely arbitrary
dependence structure in the joint distribution of
\((D,\{\Theta_k\colon k\in \mathbb{Z}^+\})\)

For availability \(n\) - vector \(y\) , define
\(A(y) = \{i\colon y^i >0\} \cup \{0\}\) to be the set of available
flights and the no- purchase option when the availability vector is
\(y\) .A customer who has preference mapping \(\theta\) and who arrives
to find availability vector \(y\) will make the choice

\[
\phi (y,\theta) = \underset {i\in A(y)}{\operatorname{argmin}}\theta (i). \tag{19}
\]

Note that \(\phi (y,\theta)\) depends on \(y\) only through \(A(y)\)
.The choice of customer \(k\) i.e., the flight on which \(k\) buys a
ticket) is \(\Phi_{k,x} = \phi (X_x(k),\Theta_k)\) ,where \(X_{x}(k)\)
is the availability vector faced by customer \(k\) when the initial
inventory allocation is \(x\) . The sequence \(\{X_{x}(k)\}\) satisfies
the recursion

\[
X_{x}(k + 1) = X_{x}(k) - \epsilon^{\Phi_{k,x}},\quad k = 1,\ldots ,D, \tag{20}
\]

with the boundary condition \(X_{x}(1) = x\) .The total sales on flight
\(i\) can be expressed as

\[
Y^{i}(x) = \sum_{k = 1}^{D}\mathbb{I}\{\Phi_{k,x} = i\} . \tag{21}
\]

The choice model complies with the basic consumer sovereignty property
(see McFadden 2000) in economics, which states that preferences are
predetermined in any choice situation, and do not depend upon the
alternatives available for selection. As mentioned in the introduction,
the setup above is equivalent to that described by Mahajan and van Ryzin
2001a,b).As shown by Mahajan and van Ryzin, it subsumes a variety of
models in the literature.

However, there are reasonable and realistic choice models that are not
covered by the above framework. For example, psychologists have shown
that when facing too many options, people may defer their decision or
search for new options (see, e.g., Iyengar and Lepper 2000). In one of
their studies (involving jars of jam), nearly \(30\%\) of customers who
are provided with a limited number of choices subsequently make a
purchase, whereas only \(3\%\) of customers who are given an extensive
number of options do so. In such a case, a customer's preference may not
be established before he observes the available choices. It should be
pointed out that the results and the solution methodology presented in
this paper do not depend upon a particular choice model; the setup
described in this section is just one case where our results apply.

\section{6.2.What Is Demand?}\label{what-is-demand}

Revenue managers often use the term unconstrained demand' to, in some
sense, mean pure demand. Owing to the interaction among multiple
flights, it is not altogether clear what these terms mean in situations
when choice behavior is present. In this section, we provide one
definition that is consistent with our earlier developments. In the
following, the demand distribution on flight \(i\) does not depend on
the seat availability on flight \(i\) and depends only on the seat
availability on all the other flights. So, what is demand when there is
choice behavior?

Define

\[
Q^{i}(x) = \sum_{k = 1}^{D}\mathbb{I}\{\phi (X_{x + \infty \in^{i}}(k),\Theta_{k}) = i\} . \tag{22}
\]

In the above, \(x + \infty \in^{i}\) is the vector \(\mathcal{X}\) with
the ith element \(x^{i}\) replaced by \(\infty\) .We call \(Q^{i}(x)\)
the demand for flight \(i\) By 21)
\(Q^{i}(x) = Y^{i}(x + \infty \epsilon^{i})\) .In other words,
\(Q^{i}(x)\) is the sales on flight \(i\) if the number of seats
allocated on flight \(i\) is infinite and the seat allocation is
\(x^{j}\) on each flight \(j\neq i\) Observe that \(Q^{i}(x)\) does not
depend upon \(x^{i}\) , but it does, however, depend upon the other
entries of \(\mathcal{X}\) via \(X_{x + \infty \in^{i}}^{j}(1) = x^{j}\)
\(j\neq i\) . Expression (22) gives us a pathwise definition, where the
random variables \(Q^{i}(x)\) are defined on the same probability space
as the process \((D,\{\Theta_k\colon k\in \mathbb{Z}^+\})\) . This
allows computation of the distributions of \(Q(x)\) through the formula

\[
F_{x}^{t}(q) = \mathrm{P}\bigg(\sum_{k = 1}^{D}\mathbb{I}\{\phi (X_{x + \infty \in^{i}}(k),\Theta_{k}) = i\} \leqslant q^{i};i = 1,\dots,n\bigg). \tag{21}
\]

The extent to which this computation is simple or not depends upon the
probabilistic assumptions we place upon the underlying space. For more
on this, see Mahajan and van Ryzin (2001a, b), who describe how many
tractable models can indeed be put into this framework.

We next demonstrate that \(Q^{i}(x)\) as defined in 22 is consistent
with our earlier use of the symbol in {\$\textbackslash S
\textbackslash S 2 - 5\$}

Specifically, we show that sales can be expressed as the minimum of
demand and the number of open seats as in expressions (1)- (2).

LEMMA 1. \(Y^{i}(x) = \min \{x^{i},Q^{i}(x)\}\) , where \(Y^{i}(x)\) is
defined by (21) and \(Q^{i}(x)\) is defined by (22).

PROOF. By (22), we have

\[
\begin{array}{rl} & Q^{i}(x) = \sum_{k = 1}^{D}\mathbb{I}\{\phi (X_{x + \infty \epsilon^{i}}(k),\Theta_{k}) = i\} \\ & \qquad = \sum_{k = 1}^{D}\big[\mathbb{I}\{\phi (X_{x + \infty \epsilon^{i}}(k),\Theta_{k}) = i\} \mathbb{I}\{X_{x}^{i}(k) > 0\} \\ & \qquad +\mathbb{I}\{\phi (X_{x + \infty \epsilon^{i}}(k),\Theta_{k}) = i\} \mathbb{I}\{X_{x}^{i}(k) = 0\} \big]\\ & \qquad = Y^{i}(x) + \sum_{k = 1}^{D}\mathbb{I}\{\phi (X_{x + \infty \epsilon^{i}}(k),\Theta_{k}) = i\} \mathbb{I}\{X_{x}^{i}(k) = 0\} . \end{array}
\]

In the above, the last equality follows because

\[
\phi (X_{x + \infty \epsilon^i}(k),\Theta_k) = \phi (X_x(k),\Theta_k)
\]

when \(X_{x}^{i}(k) > 0\) . Note also that

\[
\begin{array}{rl} & Y^{i}(x) = \sum_{k = 1}^{D}\mathbb{I}\{\phi (X_{x}(k),\Theta_{k}) = i\} \\ & \qquad = \sum_{k = 1}^{D}\mathbb{I}\{\phi (X_{x}(k),\Theta_{k}) = i\} \mathbb{I}\{X_{x}^{i}(k) > 0\} . \end{array}
\]

In (24), observe that if \(Y^{i}(x)< x^{i}\) ,then \(X_{x}^{i}(k) > 0\)
for all \(k\) Consequently, \(Y^{i}(x) = Q^{i}(x)< x^{i}\) . Similarly,
if \(Y^{i}(x) = x^{i}\) then \(Q^{i}(x)\geqslant Y^{i}(x) = x^{i}\) .
This completes the proof.

Before we proceed, it is worth pointing out there certainly could be
other reasonable definitions of demand. From the standpoint of the
models described in this paper, the key requirement is that (1)- (2)
make sense in conjunction with the definition. This also points out the
generality of the framework developed in §3, because the results there
do not depend upon the particular details that give rise to \(Q(x)\) .

We are now ready to derive upper and lower bounds on demand as needed to
apply the results of §3 to the choice model described in §6.1. Let

\[
\begin{array}{rl} & {\underline{D}^{i} = \sum_{k = 1}^{D}\mathbb{I}\{\Theta_{k}(i) = 1\} \quad \mathrm{and}}\\ & {\overline{D}^{i} = Q^{i}(0).} \end{array} \tag{26}
\]

Observe that \(D = \sum_{i = 1}^{n} \underline{D}^{i}\) . We now have
the following pathwise inequalities.

LEMMA 2.
\(\underline{D}^{i} \leqslant Q^{i}(x) \leqslant \overline{D}^{i}\) for
all \(x\) .

PROOF. First note that
\(\begin{array}{r}\overline{D}^i = Q^i (0) = \sum_{k = 1}^D\mathbb{I}\{\phi (0 + \infty \epsilon^i, \end{array}\)
\(\Theta_k) = i\} = \sum_{k = 1}^D\mathbb{I}\{\Theta_k(i)< \Theta_k(0)\}\)
.The result now follows immediately from the fact that
\(\mathbb{I}\{\Theta_k(i) = 1\} \leqslant\)
\(\mathbb{I}\{\phi (X_{x + \infty \epsilon^i},\Theta_k) = i\} \leqslant \mathbb{I}\{\Theta_k(i)< \Theta_k(0)\}\)
.

In the above, note that both \(\underline{D}^{i}\) and
\(\overline{D}^{i}\) are random variables whose distribution is
determined by exogenous choice process characteristics. We can now apply
the result of §3 to obtain bounds for the value of the MDP for the
choice models of this section. To this end, let
\(\{\underline{D}_{t} \colon t = 1, \ldots , m\}\) and
\(\{\overline{D}_{t} \colon t = 1, \ldots , m\}\) be the lower and upper
bounds for demand in periods \(t = 1, \ldots , m\) . Clearly, these
sequences satisfy the conditions needed for \(\{Q_{t}\}\) and
\(\{\overline{Q}_{t}\}\) in Proposition 2 (see, e.g., Theorem 1.2.4 of
Müller and Stoyan 2002), thereby yielding the following result.

PROPOSITION 5. For each \(i = 1, \ldots , n\) , construct an MDP for
flight \(i\) with demand sequence
\(\{\underline{D}_{t}^{i} \colon t = 1, \ldots , m\}\) (respectively,
\(\{\overline{D}_{t}^{i} \colon t = 1, \ldots , m\}\) ), and denote the
value function \(\underline{v}_{t}^{i}(s^{i})\) (respectively,
\(\overline{v}_{t}^{i}(s^{i})\) ). Then,
\(\sum_{i = 1}^{n} \underline{v}_{t}^{i}(s^{i}) \leqslant v_{t}(s) \leqslant \sum_{i = 1}^{n} \overline{v}_{t}^{i}(s^{i})\)
for \(t = 1, \ldots , m\) .

It is also evident that \(Q^{i}(x)\) , \(Y^{i}(x)\) ;
\(i = 1, \ldots , n\) and \(D\) described in §6.1 and 6.2 satisfy the
conditions needed to apply Proposition 4. Therefore, the pooled upper
bound is applicable to the choice model of §6.

The relationship between \(v_{t}^{p}\) in Proposition 4 and the upper
bound in Proposition 5 depends upon the particulars of a given problem.
To see this, note that the bound from the pooled problem is tight when
there is full substitution (i.e., when \(\Theta_{k}(0) = n + 1\) for all
\(k\) ) provided the original formulation is extended to allow such
pooling policies and the definition of the ``set of available flights''
in §6.1 is suitably modified. The upper bound in Proposition 5 is tight
when there is no substitution \((\Theta_{k}(0) = 2\) for all \(k\) ).

\section{7. Solution Approaches}\label{solution-approaches}

Because the state and action spaces are very large, exact solution of
the MDP in §2 is, for practical purposes, impossible for moderate- sized
problems. Consequently, we seek to identify heuristic methods that
perform reasonably well. One simple heuristic method for our problem is
to use the policy derived from the lower- bound booking limits. We
showed in Proposition 3 that the expected revenue from following this
policy is at least as large as the lower bound for the optimal expected
revenue \(\sum_{i = 1}^{n} \underline{v}_{m}^{i}(0)\) . In numerical
experiments, we have observed that the lower- bound booking- limit
policy outperforms other simple booking- limit policies (upper- bound
booking- limit policy, hybrid booking- limit policy from lower- and
upper- bound booking limits). So, we will use the lower- bound booking-
limit policy as a baseline in our numerical example section.

\section{7.1. Approximation Using Upper and Lower
Bounds}\label{approximation-using-upper-and-lower-bounds}

In this section, we discuss how to approximate the value function using
a weighted average of the upper and lower

bounds. The approximation scheme takes the following form:

\[
\tilde{v}_{t}(s) = \sum_{i = 1}^{n}[\beta_{i}(s,t)\bar{v}_{t}^{i}(s^{i}) + (1 - \beta_{t}(s,t))v_{t}^{i}(s^{i})]. \tag{27}
\]

Alternatively, we could use the upper bound from the pooled problem
instead of the separable upper bound. We will only discuss in detail
approximation (27).

For \(j = 1,\ldots ,n\) we have

\[
\begin{array}{rl} & {\Delta_{j}\tilde{v}_{t}(s) = \tilde{v}_{t}(s) - \tilde{v}_{t}(s + \epsilon^{j})}\\ & {\qquad = \sum_{i\neq j}(\beta_{i}(s,t) - \beta_{i}(s + \epsilon^{j},t))(\bar{v}_{t}^{i}(s^{i}) - v_{t}^{i}(s^{i}))}\\ & {\qquad +\beta_{j}(s,t)\bar{v}_{t}^{j}(s^{j}) - \beta_{j}(s + \epsilon^{j},t)\bar{v}_{t}^{j}(s^{j} + 1)}\\ & {\qquad +(1 - \beta_{j}(s,t))v_{t}^{j}(s^{j})}\\ & {\qquad -(1 - \beta_{j}(s + \epsilon^{j},t))v_{t}^{j}(s^{j} + 1).} \end{array}
\]

If we assume that \(\beta_{i}(s,t)\) and
\(\beta_{i}(s + \epsilon^{j},t)\) are very close, then the above
suggests the approximation

\[
\Delta_{j}\tilde{v}_{t}(s)\approx \beta_{j}(s,t)\Delta \tilde{v}_{t}^{j}(s^{j}) + (1 - \beta_{j}(s,t))\Delta v_{t}^{j}(s^{j}). \tag{28}
\]

When we use approximation scheme (27), expression (28) represents the
approximate marginal value of a seat on flight \(j\) as determined by a
weighted sum of the upperbound marginal value
\(\Delta \bar{v}_{t}^{j}(s^{j})\) and lower- bound marginal value
\(\Delta \nu_{t}^{j}(s^{j})\)

The choice of \(\beta_{i}(s,t)\) will greatly affect the performance of
our approximation. In our numerical experiments, constant weights,
independent of state and time, work reasonably well. We will return to
this issue in {\$\textbackslash S 9\$}

\section{7.2. Static Booking-Limit Policy from Approximate Marginal
Value}\label{static-booking-limit-policy-from-approximate-marginal-value}

In (28), if we choose constant weight, then
\(\Delta_{j}\tilde{v}_{t}(s)\) does not depend on the value of \(s^i\)
for \(i\neq j\) .In this case, let

\[
\Delta \tilde{v}_t^j (s^j)\equiv \Delta_j\tilde{v}_t(s) = \beta \Delta t_t^{j^j}(s^j) + (1 - \beta)\Delta v_t^j (s^j). \tag{29}
\]

A booking limit for flight \(j\) in period \(t,b_{t}^{j}\) can be
determined by

\[
b_{t}^{j} = \min \{0\leqslant s^{j}\leqslant c^{j}\colon \Delta \tilde{v}_{t - 1}^{j}(s^{j}) > f_{t}\} . \tag{30}
\]

The method for determining booking limits in (30) is motivated by the
results from single- flight revenue management models with no choice
behavior. Note that absent customer- choice behavior, the marginal value
given by (29) is exact, and (30) determines optimal booking limits (see
Lautenbacher and Stidham 1999). With customer choice, choosing a
``good'' weight \(\beta\) is crucial for determining good booking
limits. Alternatively, we can employ a heuristic search procedure by
varying the weight \(\beta\) computing the corresponding booking limits
according to (30), then evaluating (via simulation) the resulting
policies, and keeping the best one. The method is viable because the
simulation step usually can be done very quickly even for big problems.

The algorithm can be summarized as follows:

\begin{enumerate}
\def\labelenumi{\arabic{enumi}.}
\tightlist
\item
  Initialize: set \(b^{*} = 0\) \(v^{*} = 0\)\\
\item
  Fix \(0\leq \beta \leq \bar{\beta}\leq 1\) and \(\delta >0\) .For
  \(\beta = \beta\) to \(\bar{\beta}\) with step size \(\delta\) do the
  following: a For \(j = 1,\ldots ,n\) and \(t = 0,\ldots ,m - 1\) ,
  calculate the approximate marginal values
  \(\Delta \tilde{v}_{t}^{j}(\cdot)\) using (29). b For
  \(j = 1,\ldots ,n\) and \(t = 1,\ldots ,m\) , determine the booking
  limit on flight \(j\) in period \(t\) by
\end{enumerate}

\[
b_{t}^{j} = \min \{0\leqslant s^{j}\leqslant c^{j}\colon \Delta \tilde{v}_{t - 1}^{j}(s^{j}) > f_{t}\} .
\]

\begin{enumerate}
\def\labelenumi{\alph{enumi}.}
\setcounter{enumi}{2}
\item
  Let \(\pi^b\) be the static booking-limit policy where
  \((\pi^b)_i^j (s) = (b_t^j -s^j)^+\) . Simulate the policy \(\pi^b\)
  for \(l\) replications and record the average total revenue from all
  the flights as \(\hat{v}\) .Then, \(\hat{v}\) is an estimator for
  \(u_{m}^{\pi^b}(0)\)
\item
  If \(\hat{v} >v^{*}\) ,then \(v^{*} = \hat{v}\) and \(b^{*} = b\)
\end{enumerate}

The output of the algorithm is a booking- limit matrix \(b^{*}\) which
is used to implement a static booking- limit policy. Note also that the
algorithm is run ``offline'' before the start of the booking horizon.

In the simplest form of the algorithm, we have used lower and upper
limits \(\beta = 0\) and \(\tilde{B} = 1\) with a step size \(\delta\)
between 0.01 and 0.05. The range of the weight may be reduced by
preprocessing. For instance, one preprocessing method is to first run
the algorithm with large \(\delta\) (say \(\delta = 0.1)\) and observe
the best weight \(\tilde{\beta}\) .Subsequently, a small step size can
then be used to rerun the algorithm around a small range that contains
\(\tilde{\beta}\)

\section{7.3. Dynamic Booking-Limit Policy from Approximate Marginal
Value}\label{dynamic-booking-limit-policy-from-approximate-marginal-value}

The algorithm presented in {\$\textbackslash S 7.2\$} can also be used
to derive a dynamic booking- limit policy if the algorithm is applied
``on the fly'' to the remaining \(t\) - period problem upon entry into
each time period \(t\) .In this case, the weight is adjusted upon entry
into each time period. This version of the method takes into account the
state and time, and therefore has the potential of generating a better
control policy. At the beginning of each time \(t\) in the booking
process, observe the state \(s_t\) , and run the algorithm in
{\$\textbackslash S 7.2\$} on the remaining \(t\) - period problem. The
output of the algorithm is used to control bookings in period \(t\)
according to \(x_{t} = (b_{t}^{*} - s_{t})^{+}\)

Observe that for each \(t\) the algorithm chooses an action for period
\(t\) based upon the idea that it will use a static booking- limit
policy from that time forward. However, this is not the case, because in
subsequent periods the algorithm is run again. This is consistent with
the basic philosophy of many real- world revenue management systems in
which actions are selected by the repeated resolving of a particular
formulation throughout the booking process.

It is possible to combine value function approximation and optimization
techniques to compute actions on the fly in other ways as well. At the
beginning of each period \(t\) if we are in state \(s\) , we consider
the following optimization

problem:

\[
\max_{0\leqslant x\leqslant c - s}\operatorname {E}[r_t(x,Q_t(x)) + \tilde{v}_{t - 1}(\min \{x,Q_t(x)\} +s)], \tag{31}
\]

where \(\tilde{v}_{t - 1}(\cdot)\) could be (27) or any other
approximation. The optimization problem in (31) can be solved
approximately via heuristic search or simulation- based optimization.
The resulting action can then be used to control the booking process in
period \(t\)

The method, however, performed poorly in most of our test problems (we
do not report the numerical results here). The solution to (31) only
provides an action for the current period, and thus fails to take into
account the impact of future actions on value function approximation.
This is different from the algorithm in {\$\textbackslash S 7.2\$} which
does take into account the impact of future actions through the
simulation procedure that iterates through values of \(\beta\) As a
result, the policy suggested by the method is, in a sense, myopic; the
approximation error is accumulated period by period, and eventually
leads to the large performance gaps observed in our numerical study.
Furthermore, the method cannot be applied to certain situations where
our model assumptions are violated. For instance, if the assumption that
different fare classes arrive in distinct time periods is violated, it
is necessary to set a booking limit for each fare class instead of just
setting a booking limit for the current booking class see,e.g.Belobaba
1989,p.184.

\section{8. Deterministic Linear Programming
Formulation}\label{deterministic-linear-programming-formulation}

One widely used solution approach in revenue management is deterministic
linear programming. For an \(m\) - period problem with \(n\) single- leg
flights and no choice behavior, one would need to solve \(n\)
independent linear programming problems to get a seat allocation on each
flight. For comparison purposes, we now write the \(n\) optimization
problems in one formulation. Let \(y_{j}^{i}\) be the number of seats
allocated in period \(j\) on flight \(i\) Recall that in our block-
demand setting, time period and ticket class are synonymous.) Let
\(\mu_j^i = \mathrm{E}\underline{D}_j^i\) (hereafter we will work with
the choice model of {\$\textbackslash S 6)\$} . The linear program for
state \(s\) in period \(t\) is

\[
\begin{array}{rl} & {\max_{y}\sum_{i = 1}^{n}\sum_{j = 1}^{t}f_{j}y_{j}^{i}}\\ & {\mathrm{s.t.}\sum_{j = 1}^{t}y_{j}^{i}\leqslant c^{i} - s^{i},\quad i = 1,\dots,n,}\\ & {\quad y_{j}^{i}\leqslant \mu_{j}^{i},\quad j = 1,\dots,t;i = 1,\dots,n,} \end{array} \tag{32}
\]

where constraint (33) ensures that the total number of seats allocated
on each flight does not exceed the available capacity, and constraint
(34) stipulates that the number of seats allocated to each class on each
flight does not exceed the expected demand.Note that this is a standard
formulation that is described in, e.g., Cooper (2002).

Next, we incorporate the choice effects in the linear programming model
by considering a family of constraints parameterized by
\(\beta \in [0,1]\) . The idea is to approximate the expected demand by
a weighted sum of the upper- and lower- bound expected demand. Let
\(\begin{array}{r}\mu_j = \mathrm{ED}_j = \sum_{i = 1}^n\mu_j^i \end{array}\)
and \(\lambda_j^i = \mu_j\mathrm{P}(\Theta (i)< \Theta (0))\) . Note
that here \(\lambda_j^i\) is the expected value of the upper- bound
demand in (26). For state \(s\) and time \(t\) we replace the demand
constraint (34) with the following two constraints:

\[
\begin{array}{rl} & y_j^i\leqslant \beta \mu_j^i +(1 - \beta)\lambda_j^i,\quad j = 1,\ldots ,t;i = 1,\ldots ,n,\\ & \sum_{i = 1}^n y_j^i\leqslant \mu_j,\quad j = 1,\ldots ,t. \end{array} \tag{35}
\]

Constraint (35) requires that the number of seats allocated does not
exceed the expected demand approximated by a weighted sum of upper- and
lower- bound expected demand, and constraint (36) ensures that the total
number of seats allocated to each class does not exceed the total
expected demand in that class. Observe that if there is no choice
behavior among the flights, then (35)- (36) is equivalent to (34),
because in that case \(\lambda_j^i = \mu_j^i\) , and hence constraint
(36) is simply the sum of the constraints in (35). Note that regardless
of \(\beta\) the optimal objective function value is, in general, not
the revenue associated with implementing the policy specified by the
optimal solution of the linear program. Hereafter, for ease of
reference, we call the linear program defined by (32)- (33) and (35)-
(36) LPC.

\section{8.1. Booking-Limit Policies Derived from the Seat Allocation
Policy}\label{booking-limit-policies-derived-from-the-seat-allocation-policy}

The deterministic linear programming method divides seats on each flight
so that the seats allocated to one booking class will not be available
to other booking classes. Hence, it is possible that a higher booking
class is closed while a lower booking class is still open. In
particular, in our setting, this means seats allocated for a certain
period, if not purchased in the period in question, will be left empty
in the remaining periods. Under the reasonable assumption that the total
number of bookings is increasing in the available inventory, i.e., when
\(\textstyle \sum_{i = 1}^{n}Q_{t}^{i}(x)\) is increasing in \(x\) , it
is straightforward to show that higher expected revenue can be achieved
by allowing the remaining booking classes to book the empty seats left
from the previous periods. This can be done by converting a seat
allocation matrix to a booking- limit matrix as follows. Let
\(\hat{y}_j^i\) be the number of seats allocated on flight \(i\) in
period \(j\) that is, \(\{\hat{y}_j^i\}\) is an optimal solution to LPC.
Recall that time is counted backwards. Let \(b\) be an \(n\times t\)
booking- limit matrix whose \((i,j)\mathrm{th}\) element \(b_{j}^{i}\)
is the booking limit in period \(j\) on flight \(i\) defined as

\[
\begin{array}{rl} & b_j^i = \sum_{k = j}^t\hat{y}_k^i,\quad j = 2,\dots,t,\\ & b_1^i = c^i -s^i. \end{array} \tag{37}
\]

In the above, (38) ensures that the booking limits in the last period
(Period 1) are equal to the remaining capacity. In our numerical
procedures, we iterate through several values of \(\beta\) (as in the
algorithm of {\$\textbackslash S 7.2\$} ), and evaluate via simulation
the booking- limit policies implied by (37)- (38) and the respective
solutions of LPC. We then keep the best one. We call the policy that
results from this procedure LP. More formally, we obtain the policy LP
by replacing Steps 2(a) and 2(b) of the algorithm in {\$\textbackslash S
7.2\$} by

\(2(\mathrm{a}^{\prime}) - (\mathrm{b}^{\prime})\) Determine a booking-
limit matrix \(b\) by solving LPC and using (37)- (38).

The remainder of the algorithm remains the same.

\section{8.2. Bid Pricing}\label{bid-pricing}

Bid pricing is a revenue management method where threshold values (bid
prices) are set for capacity on each flight, and a seat is sold only if
the offered price is higher than the bid price. Talluri and van Ryzin
(1998) discuss several different methods for producing bid prices. In
our model, a set of bid prices can be produced from LPC. In particular,
the bid prices are the shadow prices associated with the capacity
constraint (33). Let \(p_t(s) = (p_t^i (s),\dots,p_t^n (s))\) be an
\(n\) - vector, where \(p_t^i (s)\) is the shadow price of capacity on
flight \(i\) given the state is \(s\) in period \(t\) . A bid- price
control policy specifies whether bookings should be accepted or denied
in each period for each state. Let
\(u_{t}(s) = (u_{t}^{1}(s),\dots,u_{t}^{n}(s))\) denote this decision,
where \(u_{t}^{i}(s) = 1\) if bookings are accepted on flight \(i\) in
period \(t\) , and \(u_{t}^{i}(s) = 0\) if bookings are rejected, so

\[
u_{t}^{i}(s) = \left\{ \begin{array}{ll}1 & \mathrm{if} f_{t}\geqslant p_{t}^{i}(s),\\ 0 & \mathrm{otherwise}. \end{array} \right.
\]

After iterating through \(\beta\) as in the procedure to obtain LP, we
will call the resulting best bid- price policy BP. Observe that,
strictly speaking, formulation (5) does not allow bid- price policies,
because the distribution of \(Q_{t}(x)\) is assumed to be induced by an
availability vector \(x\) . However, for choice processes as in
{\$\textbackslash S 6\$} , such a bid- price policy does make sense.

\section{8.3. Re-solving}\label{re-solving}

The linear program LPC can be re- solved each time period to take into
account the adjustments in capacity and expected future demand. That is,
prior to each re- solve, the capacity is decremented by the space taken
up by booked customers, and the expected demand is replaced by the
expected future demand. In the numerical experiments in
{\$\textbackslash S 9\$} , we consider re- solving versions of LP and
BP. Re- solving for bid prices involves replacing the shadow prices with
those of the reduced problem at the re- solving time.

At each re- solve point, we iterate through values of \(\beta\) again
evaluating the implied policies by simulation, and keeping the best one.
In addition, the (adjusted) solution before re- solving is also included
in the set of alternative solutions. That is, the booking limits
(respectively, bid prices) considered in re- solving include those
corresponding to different \(\beta\) values as well as the booking
limits (respectively, bid prices) from the previous period adjusted by
the most recent bookings.
